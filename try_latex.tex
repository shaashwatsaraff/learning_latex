\documentclass{article}

\usepackage[backend=biber, style=vancouver]{biblatex}
\addbibresource{MNE project SKN.bib}

\usepackage{graphicx}

\usepackage[]{blindtext}

\author{Shaashwat Saraff}

\title{My first {\LaTeX} document}
% Put "\LaTeX" in {...} to safely call the function without any arguments. Skipping the {...} would lead to formatting error - no space between 'LaTeX' and 'document'.

% Explicitly specifying date below. Skip this to auto-set date to date of compilation.
\date{15-Jun-2021}

\begin{document}

  \maketitle


  \section{How paragraphs/linefeeds work \label{paragraph_sec}}

    This is some text.
    This is a new line in my .tex, but a single linefeed doesn't break paragraph.

    Two linefeeds in the .tex breaks the paragraph.

    Also, paragraphs are automatically indented. 


  \section{Sections \label{sections_sec}}

    Sections are automatically numbered.

    You can also have subsections, subsubsections, etc.


  \section{Formatting \label{formatting_sec}}

    This is normal text. 
    \textbf{This is bold text.} 
    \textit{This is italic text.} 
    \emph{This is emphatic text.} 
    \underline{This is underlined text.}
    % textit and emph look the same by default but are stylistically different. Use textit for stuff that's always italicized like book titles, latin phrases, etc. Use epmh for anything emphatic. You can redefine what emph does to emphasise text in ways other than italicizing. 


    \subsection{Subsection example \label{reflab_subsections_sec}}

      "This is in quotation marks." If you quote like this, the " character will just be rendered as closing quote on both opening and closing sides. 

      To use quotes the right way, use the backtick ` character:
      ``This is in proper quotation marks.'' `This is how you do single quotes.'


  \section{Lists \label{lists_sec}}

    A numbered list:
    \begin{enumerate}
      % environment for numbered lists
      \item Bread
      \item Butter
      \item Toilet paper
      % Doing a sublist
        \begin{enumerate}
          \item Yellow TP
          \item Blue TP \label{ref_blue_tp}
        \end{enumerate}
      \item Milk
    \end{enumerate}

    A non-numbered list:
    \begin{itemize}
      \item Bread
      \item Butter
      \item Milk
        \begin{itemize}
          \item Whole milk 
          \item skimmed milk
        \end{itemize}
    \end{itemize}
  

  \section{Labels and references}

    Here we demonstrate LaTeX's automated section number (and beyond) referencing using labels and references.

    The section on Lists is \ref{lists_sec}. The section on Subsections is \ref{reflab_subsections_sec}. The figures section is \ref{ref_sec_figures}.

    Blue TP is item number \ref{ref_blue_tp}. 

    The Arch Logo is in Figure \ref{ref_fig_archlogo}.

  
  \section{Figures \label{ref_sec_figures}}

    Here we discuss how to include figures/images in \LaTeX{} and how to refer to them in text.

    \begin{center}
      \includegraphics[width=3in, height=3in, keepaspectratio]{figures/archlinux-logo-with-text.png}
      % could also use e.g. width=0.7\textwidth to set it to 70% of whatever the width of a text line is
    \end{center}


    % including some lorem ipsum to fill text
    \blindtext

    % This time we include the image in a figure
  
    \begin{figure}[h]
      % [h] = here, wherever you declare it in the text 
      % [p] = fig on its own page; [t]: top of page; [b]: bottom of page
      \begin{center}
        \includegraphics[width=0.5\textwidth]{figures/archlinux-logo-with-text.png}

        \caption{Arch Linux logo \label{ref_fig_archlogo}}
      \end{center}
    \end{figure}

  
  \section{Example place where bibliography references are used}

  % textcite to do in-text citations naming the author and providing a ref num
  According to \textcite{catic_aerosol-jet_2020}, AJP works fine. 

  % parencite to just cite with a number at the end of the sentence, without naming author
  Flexdym is a block copolymer. \parencite{lachaux_thermoplastic_2017}

  \section{References (bibliography)}

    \printbibliography
    
\end{document}
